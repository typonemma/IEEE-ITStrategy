\documentclass[conference]{IEEEtran}
\IEEEoverridecommandlockouts
% The preceding line is only needed to identify funding in the first footnote. If that is unneeded, please comment it out.
\usepackage{cite}
\usepackage{amsmath,amssymb,amsfonts}
\usepackage{algorithmic}
\usepackage{graphicx}
\usepackage{textcomp}
\usepackage{xcolor}
\usepackage{mdwlist}
\def\BibTeX{{\rm B\kern-.05em{\sc i\kern-.025em b}\kern-.08em
    T\kern-.1667em\lower.7ex\hbox{E}\kern-.125emX}}
\begin{document}

\title{WFO into WFH or WFA with Balanced Scorecard (BSC) Framework in Jakarta Company\\
{ \textsuperscript{}}
}

\author{\IEEEauthorblockN{1\textsuperscript{st}Glenny Chudra}
\IEEEauthorblockA{\textit{Informatics Departerment} \\
\textit{Pradita University)}\\
Tangerang, Indonesia \\
glenny.chudra.s2@student.pradita.ac.id}
\and
\IEEEauthorblockN{2\textsuperscript{nd} Alfa Yohannis}
\IEEEauthorblockA{\textit{Informatics Departement} \\
\textit{Pradita University}\\
Tangerang, Indonesia\\
alfa.yohannis@pradita.ac.id}


}

\maketitle

\begin{abstract}
an IT strategy for succeeding form work from office (WFO) into remote work (WFH) and work from anywhere (WFA) is becoming increasingly important in today's digital era. The right IT strategy can enable companies to achieve maximum productivity, increase efficiency, and ensure data security. To achieve these goals, the IT strategy must include several essential elements. First, the reliable and secure network infrastructure must be available to ensure stable and secure remote connections to workers' devices. Second, collaboration and communication platforms that enable workers to work together and communicate easily must be available and integrated with the company's systems. In addition, data protection and information security must also be the IT strategy's main focus. Companies must ensure that all devices workers use, including personal devices, are encrypted and protected from cyber-attacks. Efforts must be made to ensure the security of data accessed and exchanged by workers across platforms and the company's systems. Finally, the IT strategy must include training and technical support for workers facing technical or security challenges. Proper training can help workers understand and effectively use the tools and technology used in remote work, while technical support can help them quickly and efficiently resolve technical issues. By implementing the right IT strategy, companies can ensure that they can work effectively and securely from anywhere, enabling them to achieve their business goals efficiently and keep up with current workforce trends.

\end{abstract}

\begin{IEEEkeywords}
IT Strategy, Work From Anywhere, Work From Home, Balanced Scorecard
\end{IEEEkeywords}

\section{Introduction}
The development of information technology has changed the way we work and interact in our daily lives. In recent years, remote work has become increasingly popular, especially due to the global pandemic that has prompted many companies to adopt work from home (WFH) policies to ensure the health of their employees.

However, remote work is not the only option available to the workforce today. Some companies choose to return to the office and implement work from office (WFO) policies, while others choose to implement work from anywhere (WFA) policies, allowing employees to work from any location they desire.

In the face of these various work policies, the right information technology (IT) strategy is becoming increasingly important. The right IT strategy can help companies achieve maximum productivity, improve efficiency, and ensure data security.

In this paper, we will discuss effective IT strategies for success in various work policies, including WFH, WFO, and WFA Using Balanced Scorecard (BSC) Framework. We will discuss the key elements that must be considered in an IT strategy, including network infrastructure, collaboration and communication platforms, data security, and technical training and support for employees. By considering these elements and implementing the right IT strategy, companies can achieve success in various work policies and ensure long-term business success.

\section{Research Methods}

In this research using a qualitative research study, a questionnaire was distributed to 20 individuals with the following questions:

\section{Pembahasan dan Isi}

\subsection{Quesioneere Respondent Answer}
\begin{itemize}
\item {Work Form Home (WFH)}
\begin{enumerate}
\item{Positive aspects of WFH related to work}
\begin{enumerate}
\item Higher time flexibility without transportation costs to the office.
\item Easier focus on tasks due to minimal disturbance and distraction.
\item Ability to improve the use of information technology.
\end{enumerate}
\item{Negative aspects of WFH related to work}
\begin{enumerate}
\item Difficulty communicating if not accustomed to communicating through telecommunications media.
\item Disturbances from the surrounding environment and children who cannot follow WFH
\end{enumerate}
\end{enumerate}
\item {Work from Office (WFO)}
\begin{enumerate}
\item {Positive aspects of WFO related to work}
\begin{enumerate}
\item Ability to directly request data from authorized personnel.
\item Ability to communicate directly and socialize with co-workers.
\end{enumerate}
\item {Negative aspects of WFO related to work}
\begin{enumerate}
\item Expensive transportation costs and long travel time.
\item Distractions from the office environment.
\end{enumerate}
\end{enumerate}
\item {Work from Anywhere (WFA)}
\begin{enumerate}
\item {Positive aspects of WFA related to work}
\begin{enumerate}
\item Ability to work from anywhere and anytime.
\item Ability to choose a more productive and inspiring location.
\end{enumerate}
\item {Negative aspects of WFA related to work}
\begin{enumerate}
\item Vulnerability to technical and communication issues if not accustomed to communicating through telecommunications media.
\item Not flexible in terms of time and place for certain tasks.
\end{enumerate}
\item Challenges of WFA related to work
\begin{enumerate}
\item Difficulty interacting with family, children distracted by online games or other sites unrelated to learning material.
\item Difficulties in dealing with damages that occur in the network or technology used.
\item Difficulty concentrating when work is imbalanced and when looking for accurate numbers in work such as making financial reports in different places.
\end{enumerate}
\end{enumerate}
\item {Technologies for WFA}
\begin{enumerate}
\item Technologies used in WFA 
\begin{enumerate}
\item Google Meet
\item Slack
\item Clickup
\item Excel
\item DJP
\item VB
\item MySQL
\end{enumerate}
\item Technologies required to support WFA
\begin{enumerate}
\item[] Supporting applications for WFA are easy-to-use and easy-to-integrate team management applications, project management systems, applications for automatic report creation. In addition, reliable and fast networks are also needed to support WFA. These technologies will help individuals and teams to efficiently and effectively complete their work in working WFA.
\end{enumerate}

\end{enumerate}
\item {Mindset and Values for WFA}
\begin{enumerate}
\item Time contact limits, both between superiors and subordinates and among colleagues.
\item Determination of employee performance based on objective-based or time-based.
\item Polite tone in emails.
\item Turning on the camera during video calls.
\item Rules of neat and polite dressing.
\item Independence and self-confidence in completing work on time.
\item Respecting each other's time and being flexible in meetings or group chat
\end{enumerate}
\end{itemize}

\subsection{Applied BSC as a company strategy from WFO into WFA}





\section*{Acknowledgment}

The preferred spelling of the word ``acknowledgment'' in America is without 
an ``e'' after the ``g''. Avoid the stilted expression ``one of us (R. B. 
G.) thanks $\ldots$''. Instead, try ``R. B. G. thanks$\ldots$''. Put sponsor 
acknowledgments in the unnumbered footnote on the first page.

\section*{References}

Please number citations consecutively within brackets \cite{b1}. The 
sentence punctuation follows the bracket \cite{b2}. Refer simply to the reference 
number, as in \cite{b3}---do not use ``Ref. \cite{b3}'' or ``reference \cite{b3}'' except at 
the beginning of a sentence: ``Reference \cite{b3} was the first $\ldots$''

Number footnotes separately in superscripts. Place the actual footnote at 
the bottom of the column in which it was cited. Do not put footnotes in the 
abstract or reference list. Use letters for table footnotes.

Unless there are six authors or more give all authors' names; do not use 
``et al.''. Papers that have not been published, even if they have been 
submitted for publication, should be cited as ``unpublished'' \cite{b4}. Papers 
that have been accepted for publication should be cited as ``in press'' \cite{b5}. 
Capitalize only the first word in a paper title, except for proper nouns and 
element symbols.

For papers published in translation journals, please give the English 
citation first, followed by the original foreign-language citation \cite{b6}.

\begin{thebibliography}{00}
\bibitem{b1} G. Eason, B. Noble, and I. N. Sneddon, ``On certain integrals of Lipschitz-Hankel type involving products of Bessel functions,'' Phil. Trans. Roy. Soc. London, vol. A247, pp. 529--551, April 1955.
\bibitem{b2} J. Clerk Maxwell, A Treatise on Electricity and Magnetism, 3rd ed., vol. 2. Oxford: Clarendon, 1892, pp.68--73.
\bibitem{b3} I. S. Jacobs and C. P. Bean, ``Fine particles, thin films and exchange anisotropy,'' in Magnetism, vol. III, G. T. Rado and H. Suhl, Eds. New York: Academic, 1963, pp. 271--350.
\bibitem{b4} K. Elissa, ``Title of paper if known,'' unpublished.
\bibitem{b5} R. Nicole, ``Title of paper with only first word capitalized,'' J. Name Stand. Abbrev., in press.
\bibitem{b6} Y. Yorozu, M. Hirano, K. Oka, and Y. Tagawa, ``Electron spectroscopy studies on magneto-optical media and plastic substrate interface,'' IEEE Transl. J. Magn. Japan, vol. 2, pp. 740--741, August 1987 [Digests 9th Annual Conf. Magnetics Japan, p. 301, 1982].
\bibitem{b7} M. Young, The Technical Writer's Handbook. Mill Valley, CA: University Science, 1989.
\end{thebibliography}
\vspace{12pt}
\color{red}
IEEE conference templates contain guidance text for composing and formatting conference papers. Please ensure that all template text is removed from your conference paper prior to submission to the conference. Failure to remove the template text from your paper may result in your paper not being published.

\end{document}
