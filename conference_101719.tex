\documentclass[conference]{IEEEtran}
\IEEEoverridecommandlockouts
% The preceding line is only needed to identify funding in the first footnote. If that is unneeded, please comment it out.
\usepackage{cite}
\usepackage{amsmath,amssymb,amsfonts}
\usepackage{algorithmic}
\usepackage{graphicx}
\usepackage{textcomp}
\usepackage{xcolor}
\usepackage{mdwlist}
\usepackage{booktabs}
\usepackage{multirow}
 \usepackage{longtable}
 \usepackage{lscape}
\usepackage{caption}
\def\BibTeX{{\rm B\kern-.05em{\sc i\kern-.025em b}\kern-.08em
    T\kern-.1667em\lower.7ex\hbox{E}\kern-.125emX}}
\begin{document}

\newcommand{\al}[1]{{\textbf{\color{blue} Alfa: #1}}}

\title{WFO into WFH or WFA with Balanced Scorecard (BSC) Framework in Jakarta Company\\
{ \textsuperscript{}}
}

\author{\IEEEauthorblockN{1\textsuperscript{st}Glenny Chudra}
\IEEEauthorblockA{\textit{Informatics Departerment} \\
\textit{Pradita University}\\
Tangerang, Indonesia \\
glenny.chudra.s2@student.pradita.ac.id}
\and
\IEEEauthorblockN{2\textsuperscript{nd} Alfa Yohannis}
\IEEEauthorblockA{\textit{Informatics Departement} \\
\textit{Pradita University}\\
Tangerang, Indonesia\\
alfa.yohannis@pradita.ac.id}


}

\maketitle

\begin{abstract}
an IT strategy for succeeding form work from office (WFO) into remote work (WFH) and work from anywhere (WFA) is becoming increasingly important in today's digital era. The right IT strategy can enable companies to achieve maximum productivity, increase efficiency, and ensure data security. To achieve these goals, the IT strategy must include several essential elements. First, the reliable and secure network infrastructure must be available to ensure stable and secure remote connections to workers' devices. Second, collaboration and communication platforms that enable workers to work together and communicate easily must be available and integrated with the company's systems. In addition, data protection and information security must also be the IT strategy's main focus. Companies must ensure that all devices workers use, including personal devices, are encrypted and protected from cyber-attacks. Efforts must be made to ensure the security of data accessed and exchanged by workers across platforms and the company's systems. Finally, the IT strategy must include training and technical support for workers facing technical or security challenges. Proper training can help workers understand and effectively use the tools and technology used in remote work, while technical support can help them quickly and efficiently resolve technical issues. By implementing the right IT strategy, companies can ensure that they can work effectively and securely from anywhere, enabling them to achieve their business goals efficiently and keep up with current workforce trends.

\end{abstract}

\begin{IEEEkeywords}
IT Strategy, Work From Anywhere, Work From Home, Balanced Scorecard
\end{IEEEkeywords}

\section{Introduction}
The development of information technology has changed the way we work and interact in our daily lives. In recent years, remote work has become increasingly popular, especially due to the global pandemic that has prompted many companies to adopt work from home (WFH) policies to ensure the health of their employees.

However, remote work is not the only option available to the workforce today. Some companies choose to return to the office and implement work from office (WFO) policies, while others choose to implement work from anywhere (WFA) policies, allowing employees to work from any location they desire.

In the face of these various work policies, the right information technology (IT) strategy is becoming increasingly important. The right IT strategy can help companies achieve maximum productivity, improve efficiency, and ensure data security.

In this paper, we will discuss effective IT strategies for success in various work policies, including WFH, WFO, and WFA Using Balanced Scorecard (BSC) Framework. We will discuss the key elements that must be considered in an IT strategy, including network infrastructure, collaboration and communication platforms, data security, and technical training and support for employees. By considering these elements and implementing the right IT strategy, companies can achieve success in various work policies and ensure long-term business success.

\section{WFO, WFH, and WFA}
\al{Tambahkan pengertian WFO dan Remote working(WFH dan WFA).}.

\al{Jelaskan juga bahwa penelitian tentang plus minus WFO dan Remote Working sudah pernah dilakukan sebelumnnya. Kutip hasil penelitian yang sudah ada dan tambahkan ke 
dalam referensi daftar referensi.}

\section{Berpindah dari WFO ke Remote Working}
\al{Jelaskan bahwa berpindah dari WFO dan Remote Working atau mengadopsi pola kerja jarak jauh bukanlah hal sepele. Remote working memiliki tantangan. Dengan mengetahui kelemahan dan kebutuhan Remote Working, perusahaan dapat membuat strategi dan perencanaan organisasi dan TI untuk mengatasi kelemahan dan memenuhi kebutuhan.}

\al{Selain itu, budaya kerja di Indonesia memiliki keunikan tersendiri. Xari beberapa sumber tentang ini dan kutip dan sebutkan.}.
\al{Oleh karena itu, penting untuk mencari data baru plus minus WFO, WFH, WFA dari para pekerja di Indonesia. Diharapkan dapat memberi pencerahan baru mengenai kerja jarak jauh di Indonesia.}.

\section{Balanced Scorecard}
\al{Tambah kutipan dan daftar referensi dengan menggunakan bibtex. Bisa dilihat di makalah yang ada di Overleaf. Tambahkan juga peran dan tujuan penggunaan Balanced Scorecard di penelitian ini. Sebaiknya ganti Balanced Scorecard dengan IT Balanced Scorecard. Bisa cari buku `Implementing the IT Balanced Scorecard: Aligning IT with Corporate Strategy' di di web https://libgen.is/ .}
Balanced Scorecard (BSC) diciptakan oleh Robert Kaplan dan David Norton di tahun 1982. BSC sendiri merupakan suatu metode manajemen kinerja yang berguna untuk membantu organisasi mencapai visi dan strategi bisnis jangka panjang dengan memetrikasi kinerja bisnis pada empat perspektif yang berbeda, diantaranya adalah:
\begin{enumerate}
\item Perspektif Keuangan 
Tujuan perspektif keuangan adalah untuk mengukur keberhasilan organisasi dalam mencapai tujuan finansialnya. Contoh dari perspektif ini adalah pembahasan matrix dari laba bersih, pertumbuhan pendapatan dan pengembalian investasi.
\item Perspektif Pelanngan 
Tujuan perspektif adalah berfokus pada pelanggan dan upaya organisasi untuk memenuhi kebutuhan pelanggan. Contoh dari perspektif ini adalah tingkat kepuasan pelanggan, pangsa pasar, dan waktu respon pelanggan.
\item Perspektif proses Internal
Tujuan perspektif Proses Internal adalah digunakan untuk mencapai proses internal baik secara proses bisnis dan operasi. Contoh dari perspektif ini adalah waktu siklus proses, produktivitas, dan kualitas produk atau layanan.
\item Perspektif Pembelajaran dan Pertumbuhan
Tujuan Perspektif Pembelajaran dan Pertumbuhan adalah 
\end{enumerate}

\section{Research Methods}
\al{Jelaskan dengan singkat metode kualitatif lengkap dengan kutipan. Jelaskan juga metode qualitatif secara khusus yang dipakai adalah metode induktif. Jelaskan metode induktif lengkap dengan kutipan.} 

In this research using a qualitative research study, a questionnaire was distributed to 20 individuals with the following questions:
\al{Sebisa mungkin hindari penggunaan item-item/bullet-bullet. Sebaiknya jelaskan dalam bentukp paragraf. Misalnya melalui kuesioner, kami menanyakan bidang industri, peran/posisi kerja, dst. Selain itu, kami juga mengumpulkan pendapat responden tentang plus minus WFH ... dst. Untuk mengetahui penggunaan teknologi, kami menanyakan. dst.}
\begin{enumerate}
\item Field or Industry
\item Job Title or Role
\item Job Description
\item Benefits of WFH
\item Disadvantages of WFH
\item Benefits of WFO
\item Disadvantages of WFO
\item Benefits of WFA
\item Disadvantages of WFA
\item Technologies used during WFH or WFA
\item Technologies needed during WFH or WFA
\item Mindset during WFH or WFA
\end{enumerate}

\section{Results and Discussions}

%\subsection{Questionnaire Respondent Answer 
\begin{itemize}
\item {Work Form Home (WFH) \al{poin ini dan yang selevel dinaikkan aja jadi sub-section}}
\begin{enumerate}
\item{Positive aspects of WFH related to work}
\begin{enumerate}
\item Higher time flexibility without transportation costs to the office.
\item Easier focus on tasks due to minimal disturbance and distraction.
\item Ability to improve the use of information technology.
\end{enumerate}
\item{Negative aspects of WFH related to work}
\begin{enumerate}
\item Difficulty communicating if not accustomed to communicating through telecommunications media.
\item Disturbances from the surrounding environment and officer who cannot follow environment WFH and rules
\end{enumerate}
\item Frecuency item respondent answer in WFH :


\centering
\captionof{table}{WFH Answer Frecuency} \label{tab:title} 
\begin{table}[h]
\center
\begin{tabular}{|l|l|l|}
\hline
Area                 & Aspect                    & Question                                                                                                                                            \\ \hline
\multirow{5}{*}{WFH} & \multirow{3}{*}{Positive} & \begin{tabular}[c]{@{}l@{}}Higher time flexibility without \\ transportation costs to the office\end{tabular}                                       \\ \cline{3-3} 
                     &                           & \begin{tabular}[c]{@{}l@{}}Easier focus on tasks due to minimal \\ disturbance and distraction.\end{tabular}                                        \\ \cline{3-3} 
                     &                           & \begin{tabular}[c]{@{}l@{}}Ability to improve the use of information\\  technology\end{tabular}                                                     \\ \cline{2-3} 
                     & \multirow{2}{*}{Negative} & \begin{tabular}[c]{@{}l@{}}Difficulty communicating if not accustomed to\\  communicating \\ through telecommunications media.\end{tabular}         \\ \cline{3-3} 
                     &                           & \begin{tabular}[c]{@{}l@{}}Disturbances from the surrounding environment\\  and officer who \\ cannot follow environment WFH and rules\end{tabular} \\ \hline
\end{tabular}
\end{table}


\end{enumerate}
\item {Work from Office (WFO)}
\begin{enumerate}
\item {Positive aspects of WFO related to work}
\begin{enumerate}
\item Ability to directly request data from authorized personnel.
\item Ability to communicate directly and socialize with co-workers.
\end{enumerate}
\item {Negative aspects of WFO related to work}
\begin{enumerate}
\item Expensive transportation costs and long travel time.
\item Distractions from the office environment.
\end{enumerate}
\item Frecuency item respondent answer in WFO :

\centering
\captionof{table}{WFO Answer Frecuency} \label{tab:title} 
\begin{table}[h]
\center
\begin{tabular}{|l|l|l|r|}
\hline
Area                 & Aspect                    & Question                                                                                                  & \multicolumn{1}{l|}{Frecuency} \\ \hline
\multirow{4}{*}{WFO} & \multirow{2}{*}{Positive} & \begin{tabular}[c]{@{}l@{}}Ability to directly request data \\ authorized personnel.\end{tabular}         & 4                              \\ \cline{3-4} 
                     &                           & \begin{tabular}[c]{@{}l@{}}Ability to communicate directly\\  and socialize with co-workers.\end{tabular} & 9                              \\ \cline{2-4} 
                     & \multirow{2}{*}{Negative} & \begin{tabular}[c]{@{}l@{}}Expensive transportation costs \\ and long travel time.\end{tabular}           & 5                              \\ \cline{3-4} 
                     &                           & \begin{tabular}[c]{@{}l@{}}Distraction from the \\ office environment\end{tabular}                        & 3                              \\ \hline
\end{tabular}
\end{table}



\end{enumerate}
\item {Work from Anywhere (WFA)}
\begin{enumerate}
\item {Positive aspects of WFA related to work}
\begin{enumerate}
\item Ability to work from anywhere and anytime.
\item Ability to choose a more productive and inspiring location.


\end{enumerate}
\item {Negative aspects of WFA related to work}
\begin{enumerate}
\item Vulnerability to technical and communication issues if not accustomed to communicating through telecommunications media.
\item Not flexible in terms of time and place for certain tasks.
\end{enumerate}
\item Challenges of WFA related to work
\begin{enumerate}
\item Difficulty interacting with family, children distracted by online games or other sites unrelated to learning material.
\item Difficulties in dealing with damages that occur in the network or technology used.
\item Difficulty concentrating when work is imbalanced and when looking for accurate numbers in work such as making financial reports in different places.
\end{enumerate}
\item Frecuency item respondent answer in WFA :


\begin{table}[h]
\label{tab:wfa_plus_minus}
\caption{WFA Answer Frecuency}

\begin{tabular}{|l|l|l|}

\hline
\multirow{4}{*}{WFA} & \multirow{2}{*}{Positive} & \begin{tabular}[c]{@{}l@{}}Ability to directly \\ request data from\\ authorized personnel.\end{tabular}                                                                   \\ \cline{3-3} 
                     &                           & \begin{tabular}[c]{@{}l@{}}Kemampuan memilih\\ lokasi yang lebih \\ produktif dan inspiratif.\end{tabular}                                                                 \\ \cline{2-3} 
                     & \multirow{2}{*}{Negative} & \begin{tabular}[c]{@{}l@{}}Rawan terjadinya masalah \\ teknis dan masalah komunikasi\\  jika tidak terbiasa \\ berkomunikasi melalui \\ media telekomunikasi.\end{tabular} \\ \cline{3-3} 
                     &                           & \begin{tabular}[c]{@{}l@{}}Tidak fleksibel dalam hal \\ waktu dan tempat untuk \\ pekerjaan tertentu.\end{tabular}                                                         \\ \hline
\end{tabular}
\end{table}

\al{Sebisa mungkin tabel dijelaskan dalam bentuk paragraf, hindari dalam bentuk bullet atau poin-poin. Contoh: Tabel \ref{tab:wfa_plus_minus} menrangkum hasil responden mengenai keuntungan dan kekurangan WFA. Responden menyatakan bahwa keuntungan yang dapat ditawarkan oleh WFA adalah fleksibilitas dalam memilih lokasi yang paling produktif ... kerugiannya adalah ... dst. }


\captionof{table}{WFA Answer Frecuency 2 (Cont)} \label{tab:title} 



\begin{enumerate}
\item Technologies used in WFA

 
\begin{enumerate}
\item Google Meet
\item Slack
\item Clickup
\item Excel
\item DJP
\item VB
\item MySQL
\end{enumerate}
\item Technologies required to support WFA
\begin{enumerate}
\item[] Supporting applications for WFA are easy-to-use and easy-to-integrate team management applications, project management systems, applications for automatic report creation. In addition, reliable and fast networks are also needed to support WFA. These technologies will help individuals and teams to efficiently and effectively complete their work in working WFA.
\end{enumerate}
\item Frecuency item respondent answer to technology used or requirement :


\centering
\captionof{table}{Technology Answer Frecuency} \label{tab:title} 
\centering \begin{table}[h]
\center
\begin{tabular}{|l|l|l|r|}
\hline
Area                         & Aspect                    & Question                                                                    & \multicolumn{1}{l|}{Frecuency} \\ \hline
\multirow{11}{*}{Technology} & \multirow{7}{*}{Used}     & \begin{tabular}[c]{@{}l@{}}Google Meet/\\ Zoom/Teams\end{tabular}           & 9                              \\ \cline{3-4} 
                             &                           & Slack                                                                       & 1                              \\ \cline{3-4} 
                             &                           & Clickup                                                                     & 1                              \\ \cline{3-4} 
                             &                           & Excel/GoogleSheet                                                           & 4                              \\ \cline{3-4} 
                             &                           & \begin{tabular}[c]{@{}l@{}}DJP (Taxing and \\ Accounting Apps)\end{tabular} & 1                              \\ \cline{3-4} 
                             &                           & \begin{tabular}[c]{@{}l@{}}VB (Programming \\ Apps)\end{tabular}            & 2                              \\ \cline{3-4} 
                             &                           & \begin{tabular}[c]{@{}l@{}}MySQL (Database \\ Apps)\end{tabular}            & 2                              \\ \cline{2-4} 
                             & \multirow{4}{*}{Required} & \begin{tabular}[c]{@{}l@{}}Easy to used \\ and integrated\end{tabular}      & 5                              \\ \cline{3-4} 
                             &                           & \begin{tabular}[c]{@{}l@{}}Project Management \\ System\end{tabular}        & 1                              \\ \cline{3-4} 
                             &                           & \begin{tabular}[c]{@{}l@{}}Automatic based \\ application\end{tabular}      & 2                              \\ \cline{3-4} 
                             &                           & \begin{tabular}[c]{@{}l@{}}Reliable and \\ fast network\end{tabular}        & 3                              \\ \hline
\end{tabular}
\end{table}


\end{enumerate}
\item {Mindset and Values for WFA}
\begin{enumerate}
\item Time contact limits, both between superiors and subordinates and among colleagues.
\item Determination of employee performance based on objective-based or time-based.
\item Polite tone in emails.
\item Turning on the camera during video calls.
\item Rules of neat and polite dressing.
\item Independence and self-confidence in completing work on time.
\item Respecting each other's time and being flexible in meetings or group chat

\item Frecuency item respondent answer to technology used or requirement :


\centering
\captionof{table}{Mindset and Values for WFA} \label{tab:title} 
\centering\begin{table}[h]
\center
\begin{tabular}{|c|l|r|}
\hline
\multicolumn{1}{|l|}{Area} & Question                                                                                                                             & \multicolumn{1}{l|}{Frecuency} \\ \hline
\multirow{6}{*}{Mindset}   & \begin{tabular}[c]{@{}l@{}}Adanya batasan waktu kontak, baik \\ antara atasan dan bawahan maupun \\ sesama rekan kerja.\end{tabular} & 2                              \\ \cline{2-3} 
                           & Nada kalimat yang sopan dalam email.                                                                                                 & 4                              \\ \cline{2-3} 
                           & Menyalakan kamera pada saat video call.                                                                                              & 2                              \\ \cline{2-3} 
                           & Aturan berpakaian yang rapi dan sopan.                                                                                               & 3                              \\ \cline{2-3} 
                           & \begin{tabular}[c]{@{}l@{}}Kemandirian dan kepercayaan diri dalam \\ menyelesaikan pekerjaan tepat waktu.\end{tabular}               & 5                              \\ \cline{2-3} 
                           & \begin{tabular}[c]{@{}l@{}}Saling menghargai waktu dan fleksibel \\ dalam meeting atau sesi kerja bersama.\end{tabular}              & 6                              \\ \hline
\end{tabular}
\end{table}
\end{enumerate}
\end{enumerate}
\end{itemize}






\subsection{Applied BSC as a company strategy from WFO into WFA}

Setelah mendapatkan data respondent, kemudian akan dihitung frekuensi per data dan diklasifikasikan kedalam 4 area sebagai berikut:



\captionof{table}{Pembagian Area Balanced Scorecard} \label{tab:title} 

\begin{table}[h]
\center
 \begin{tabular}{|l|l|}
\hline
\multicolumn{1}{|c|}{Arti} & \multicolumn{1}{c|}{Area} \\ \hline
C                          & Customer                  \\ \hline
F                          & Finance                   \\ \hline
I                          & Internal                  \\ \hline
L                          & Learning and Growth       \\ \hline
\end{tabular}

\end{table}



\text Kemudian dari pembagian diatas kita kelompokan menjadi 4 area disemua bagian:


\captionof{table}{Description devided into 4 areas BSC} \label{tab:title} 

% Please add the following required packages to your document preamble:
% \usepackage{multirow}

\begin{table}[h]
\center
<<<<<<< HEAD
>>>>>>> main
=======
>>>>>>> main
\begin{tabular}{|l|l|l|r|l|}
\hline
Area                 & Aspect                      & Question                                                                                                                                                                                              & \multicolumn{1}{l|}{Fc} & Area \\ \hline
\multirow{5}{*}{WFH} & \multirow{3}{*}{Positive}   & \begin{tabular}[c]{@{}l@{}}Higher time flexibility without\\  transportation costs to the office\end{tabular}                                                                                         & 6                              & I, F \\ \cline{3-5} 
                     &                             & \begin{tabular}[c]{@{}l@{}}Easier focus on tasks due to \\ minimal disturbance and distraction.\end{tabular}                                                                                          & 2                              & L    \\ \cline{3-5} 
                     &                             & \begin{tabular}[c]{@{}l@{}}Ability to improve the use of \\ information technology\end{tabular}                                                                                                       & 4                              & L    \\ \cline{2-5} 
                     & \multirow{2}{*}{Negative}   & \begin{tabular}[c]{@{}l@{}}Difficulty communicating if not \\ accustomed to communicating \\ through telecommunications media.\end{tabular}                                                           & 8                              & I, L \\ \cline{3-5} 
                     &                             & \begin{tabular}[c]{@{}l@{}}Disturbances from the surrounding \\ environment and officer who \\ cannot follow environment WFH and rules\end{tabular}                                                   & 5                              & I    \\ \hline
\multirow{4}{*}{WFO} & \multirow{2}{*}{Positive}   & \begin{tabular}[c]{@{}l@{}}Kemampuan langsung meminta \\ data dari orang yang berwenang.\end{tabular}                                                                                                 & 4                              & I    \\ \cline{3-5} 
                     &                             & \begin{tabular}[c]{@{}l@{}}Kemampuan berkomunikasi secara\\  langsung dan sosialisasi dengan\\  rekan kerja.\end{tabular}                                                                             & 9                              & L    \\ \cline{2-5} 
                     & \multirow{2}{*}{Negative}   & \begin{tabular}[c]{@{}l@{}}Biaya transport yang mahal dan \\ waktu perjalanan yang lama.\end{tabular}                                                                                                 & 5                              & F    \\ \cline{3-5} 
                     &                             & Distraksi dari lingkungan kantor.                                                                                                                                                                     & 3                              & I    \\ \hline
\multirow{7}{*}{WFA} & \multirow{2}{*}{Positive}   & \begin{tabular}[c]{@{}l@{}}Ability to directly request \\ data from authorized personnel.\end{tabular}                                                                                                & 6                              & L    \\ \cline{3-5} 
                     &                             & \begin{tabular}[c]{@{}l@{}}Kemampuan memilih lokasi \\ yang lebih produktif dan inspiratif.\end{tabular}                                                                                              & 5                              & L    \\ \cline{2-5} 
                     & \multirow{2}{*}{Negative}   & \begin{tabular}[c]{@{}l@{}}Rawan terjadinya masalah teknis \\ dan masalah komunikasi jika tidak terbiasa \\ berkomunikasi melalui media telekomunikasi.\end{tabular}                                  & 5                              & C    \\ \cline{3-5} 
                     &                             & \begin{tabular}[c]{@{}l@{}}Tidak fleksibel dalam hal waktu dan\\  tempat untuk pekerjaan tertentu.\end{tabular}                                                                                       & 6                              & L    \\ \cline{2-5} 
                     & \multirow{3}{*}{Challanges} & \begin{tabular}[c]{@{}l@{}}Kesulitan berinteraksi dengan \\ keluarga, anak-anak teralihkan \\ dengan game online atau \\ situs-situs lain yang tidak terkait \\ dengan materi pelajaran.\end{tabular} & 3                              & L    \\ \cline{3-5} 
                     &                             & \begin{tabular}[c]{@{}l@{}}Kerusakan yang sulit diatasi jika \\ terjadi pada jaringan atau \\ teknologi yang digunakan.\end{tabular}                                                                  & 5                              & L    \\ \cline{3-5} 
                     &                             & Sulit Fokus                                                                                                                                                                                           & 3                              & L    \\ \hline
\end{tabular}
\end{table}




\section*{Acknowledgment}

The preferred spelling of the word ``acknowledgment'' in America is without 
an ``e'' after the ``g''. Avoid the stilted expression ``one of us (R. B. 
G.) thanks $\ldots$''. Instead, try ``R. B. G. thanks$\ldots$''. Put sponsor 
acknowledgments in the unnumbered footnote on the first page. Test \cite{paige2009model}.


\bibliography{references}
\bibliographystyle{IEEEtran}

\appendices
\section{Questionnaire Form}
\label{app:questionnaire_form}

\end{document}
