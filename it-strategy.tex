\documentclass[review]{elsarticle}

\usepackage{lineno,hyperref}
\modulolinenumbers[5]

\journal{Journal of \LaTeX\ Templates}

%%%%%%%%%%%%%%%%%%%%%%%
%% Elsevier bibliography styles
%%%%%%%%%%%%%%%%%%%%%%%
%% To change the style, put a % in front of the second line of the current style and
%% remove the % from the second line of the style you would like to use.
%%%%%%%%%%%%%%%%%%%%%%%

%% Numbered
%\bibliographystyle{model1-num-names}

%% Numbered without titles
%\bibliographystyle{model1a-num-names}

%% Harvard
%\bibliographystyle{model2-names.bst}\biboptions{authoryear}

%% Vancouver numbered
%\usepackage{numcompress}\bibliographystyle{model3-num-names}

%% Vancouver name/year
%\usepackage{numcompress}\bibliographystyle{model4-names}\biboptions{authoryear}

%% APA style
%\bibliographystyle{model5-names}\biboptions{authoryear}

%% AMA style
%\usepackage{numcompress}\bibliographystyle{model6-num-names}

%% `Elsevier LaTeX' style
\bibliographystyle{elsarticle-num}
%%%%%%%%%%%%%%%%%%%%%%%

\begin{document}

\begin{frontmatter}

\title{Jurnal WFX }
\tnotetext[mytitlenote]{Fully documented templates are available in the elsarticle package on \href{http://www.ctan.org/tex-archive/macros/latex/contrib/elsarticle}{CTAN}.}

%% Group authors per affiliation:
\author{Glenny\fnref   {myfootnote}}
\address{Pradita University, Kelapa Dua, Tangerang}




\begin{abstract}
an IT strategy for succeeding in remote work (WFH), office work (WFO), and work from anywhere (WFA) is becoming increasingly important in today's digital era. The right IT strategy can enable companies to achieve maximum productivity, increase efficiency, and ensure data security. To achieve these goals, the IT strategy must include several essential elements. First, the reliable and secure network infrastructure must be available to ensure stable and secure remote connections to workers' devices. Second, collaboration and communication platforms that enable workers to work together and communicate easily must be available and integrated with the company's systems. In addition, data protection and information security must also be the IT strategy's main focus. Companies must ensure that all devices workers use, including personal devices, are encrypted and protected from cyber-attacks. Efforts must be made to ensure the security of data accessed and exchanged by workers across platforms and the company's systems. Finally, the IT strategy must include training and technical support for workers facing technical or security challenges. Proper training can help workers understand and effectively use the tools and technology used in remote work, while technical support can help them quickly and efficiently resolve technical issues. By implementing the right IT strategy, companies can ensure that they can work effectively and securely from anywhere, enabling them to achieve their business goals efficiently and keep up with current workforce trends.

\end{abstract}

\begin{keyword}
\emph{IT Strategy \sep Work From Home \sep Work From Anywhere. }
\end{keyword}

\end{frontmatter}

\linenumbers

\section{Pendahuluan}

Perkembangan teknologi informasi telah mengubah cara kita bekerja dan berinteraksi dalam kehidupan sehari-hari. Dalam beberapa tahun terakhir, bekerja dari jarak jauh telah menjadi semakin populer, terutama karena adanya pandemi global yang mendorong banyak perusahaan untuk mengadopsi kebijakan bekerja dari rumah (WFH) untuk menjaga kesehatan karyawan mereka.

Namun, pekerjaan dari jarak jauh bukanlah satu-satunya opsi yang tersedia untuk tenaga kerja saat ini. Beberapa perusahaan memilih untuk kembali ke kantor dan menerapkan kebijakan bekerja dari kantor (WFO), sementara yang lain memilih untuk menerapkan kebijakan bekerja dari mana saja (WFA), yang memungkinkan karyawan bekerja dari lokasi mana pun yang mereka inginkan.

Dalam menghadapi berbagai jenis kebijakan kerja ini, strategi teknologi informasi (IT) yang tepat menjadi semakin penting. Strategi IT yang tepat dapat membantu perusahaan mencapai produktivitas maksimum, meningkatkan efisiensi, dan memastikan keamanan data.

Pada jurnal ini akan dibahas seputar strategi IT yang efektif untuk suksesi dalam berbagai kebijakan kerja, termasuk WFH, WFO, dan WFA. Kami akan membahas elemen-elemen kunci yang harus dipertimbangkan dalam strategi IT, termasuk infrastruktur jaringan, platform kolaborasi dan komunikasi, keamanan data, dan pelatihan dan dukungan teknis untuk karyawan. Dengan mempertimbangkan elemen-elemen ini dan mengimplementasikan strategi IT yang tepat, perusahaan dapat mencapai kesuksesan dalam berbagai jenis kebijakan kerja dan memastikan keberhasilan bisnis jangka panjang.




\section{Metode Penelitian}

Pada penelitian menggunakan metode kualitatif dengan menyebarkan kuesioner sebanyak 12 orang dengan pertanyaan sebagai berikut : 
\begin{enumerate}[1]
\item Bidang atau Jenis Industri
\item Jabatan atau Peran
\item Deskripsi Pekerjaan
\item Keuntungan WFH
\item Kerugian WFH
\item Keuntungan WFO
\item Keruagian WFO
\item Keuntungan WFA
\item Kerugian WFA
\item Teknologi yang digunakan semasa WFH ataupun WFA
\item Teknologi yang dibntuhkan semasa WFH ataupun WFA
\item Mindset pada saat WFH ataupun WFA
\end{enumerate}


\section{Hasil dan Pembahasan}

Berdasarkan kuesioner yang telah dibagikan, memioliki kesimpulan sebagai berikut: 
\begin{itemize}
\item Keuntungan-keuntungan/sisi POSITIF dari Work from HOME (WFH) berkaitan dengan pekerjaan:
\begin{itemize}
\item Fleksibilitas waktu yang lebih tinggi, tanpa biaya transport ke kantor.
\item Memudahkan fokus pada tugas karena minimnya gangguan dan distraksi.
\item Kemampuan meningkatkan penggunaan teknologi informasi.
\item Kekurangan-kekurangan/sisi NEGATIF dari Work from HOME (WFH) berkaitan dengan pekerjaan:
\item Sulit berkomunikasi jika tidak terbiasa berkomunikasi melalui media telekomunikasi.
\item Gangguan dari lingkungan sekitar dan anak-anak yang tidak bisa mengikuti aturan WFH.
\end{itemize}
\item Keuntungan-keuntungan/sisi POSITIF dari Work from OFFICE (WFO) berkaitan dengan pekerjaan:
\begin{itemize}
\item Kemampuan langsung meminta data dari orang yang berwenang.
\item Kemampuan berkomunikasi secara langsung dan sosialisasi dengan rekan kerja.
\item Kekurangan-kekurangan/sisi NEGATIF dari Work from OFFICE (WFO) berkaitan dengan pekerjaan:
\item Biaya transport yang mahal dan waktu perjalanan yang lama.
\item Distraksi dari lingkungan kantor.
\end{itemize}
\item Keuntungan-keuntungan/sisi POSITIF dari Work from ANYWHERE (WFA) berkaitan dengan pekerjaan:
\begin{itemize}
\item Kemampuan bekerja dari mana saja dan kapan saja.
\item Kemampuan memilih lokasi yang lebih produktif dan inspiratif.
\item Kekurangan-kekurangan/sisi NEGATIF dari Work from ANYWHERE (WFA) berkaitan dengan pekerjaan:
\item Rawan terjadinya masalah teknis dan masalah komunikasi jika tidak terbiasa berkomunikasi melalui media telekomunikasi.
\item Tidak fleksibel dalam hal waktu dan tempat untuk pekerjaan tertentu.
\item Kekurangan/Sisi Negatif Work from Anywhere (WFA):
\end{itemize}
\item Beberapa kekurangan/sisi negatif Work from Anywhere (WFA) yang mungkin dihadapi antara lain:
\begin{itemize}
\item Kesulitan berinteraksi dengan keluarga, anak-anak teralihkan dengan game online atau situs-situs lain yang tidak terkait dengan materi pelajaran.
\item Kerusakan yang sulit diatasi jika terjadi pada jaringan atau teknologi yang digunakan.
\item Kesulitan konsentrasi saat pekerjaan tidak seimbang dan saat mencari angka yang tepat dalam pekerjaan seperti membuat laporan keuangan di tempat yang berbeda
\end{itemize}
\item Teknologi yang digunakan dalam Work from Anywhere (WFA):
\begin{itemize}
\item Google Meet
\item Slack
\item Clickup
\item Excel
\item DJP
\item VB
\item MySQL
\end{itemize}
\item Teknologi yang dibutuhkan untuk mendukung Work from Anywhere (WFA):

Aplikasi yang mendukung untuk WFA adalah aplikasi manajemen tim yang mudah digunakan dan mudah diintegrasi, sistem manajemen proyek, aplikasi untuk pembuatan laporan otomatis. selain aplikasi untuk menunjang WFA juga dibutuhkan jaringan andal dan cepat. Teknologi-teknologi tersebut akan membantu individu dan tim dalam menyelesaikan pekerjaan mereka secara efisien dan efektif dalam bekerja secara WFA. 

\item Mindset, nilai, prinsip, etika, norma, etiket, aturan, atau mekanisme (proses) yang diperlukan agar Work from Anywhere (WFA) dapat sukses:
\begin{itemize}
\item Adanya batasan waktu kontak, baik antara atasan dan bawahan maupun sesama rekan kerja.
\item Penentuan performa karyawan berdasarkan objective-based atau time-based.
\item Nada kalimat yang sopan dalam email.
\item Menyalakan kamera pada saat video call.
\item Aturan berpakaian yang rapi dan sopan.
\item Kemandirian dan kepercayaan diri dalam menyelesaikan pekerjaan tepat waktu.
\item Saling menghargai waktu dan fleksibel dalam meeting atau sesi kerja bersama.

\end{itemize}
\end{itemize}

\section{Kesimpulan}




\section*{References}



\end{document}